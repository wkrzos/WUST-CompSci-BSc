\section{Introduction}
In a dynamic market environment influenced by a conflict to the east of Poland, increased immigration, and government policies, flat rental prices have become a significant concern for the residents of major Polish cities. This paper examines rental prices in six major Polish cities most affected by the current state of affairs: Warsaw, Kraków, Łódź, Wrocław, Poznań, and Gdańsk.

\subsection{Background of the Polish Flat Rental Market}
Poland's rental market has undergone major changes after its post-communist transition - widely recognised to have started in 1989 \cite{noauthor_25_2014}. Urbanisation and demographic shifts have greatly influenced the economy in the past years, starting in 2019, including the rental market \cite{ouanes_chapter_nodate}. The demand for rental properties has increased, especially in major cities, driven by factors i.e. job opportunities, education possibilities, and lifestyle preferences \cite{kazmierczak_korekta_2024}. The six major cities are recognised to be the economic and cultural hubs, attracting a diverse population. That increased the population of professionals, students, and expatriates, who all contribute to the current state of the rental market. An understanding of the current market climate is crucial for real estate agencies and potential renters, to make informed decisions.

\subsection{Goal of the Paper}
The paper's goal is to provide a comprehensive analysis the current rental properties market in six major Polish cities. By comparing these cities, the paper aims to identify trends, disparities, and underlying factors influencing rental costs.

\subsection{Research Questions}
Considering the above, it has been decided to form the following research questions:
\begin{itemize}
    \item \textbf{Research Question 1:} What are the current flat rental prices in Warsaw, Kraków, Łódź, Wrocław, Poznań, and Gdańsk?
    \item \textbf{Research Question 2:} How are rental properties spaced in a city's borders?
    \item \textbf{Research Question 3:} What factors contribute to the differences in rental prices across cities and to what extent can the rent prices be rpedicted using said factors?
\end{itemize}
By answering these questions, the paper seeks to uncover the key determinants of rental price variations and provide a detailed understanding of the rental landscape in these urban areas.

\subsection{Structure of the Paper}
The paper is structured as follows:
\begin{itemize}
    \item \textbf{Section 2: Methodology} - Outlines the data collection methods and analytical techniques used to assess rental prices.
    \item \textbf{Section 3: Results} - The findings of the analysis, including comparative rental prices and trends across the six cities, are presented in this section.
    \item \textbf{Section 4: Discussion} - Interprets the results, discussing the factors influencing rental prices and the implications for different stakeholders.
    \item \textbf{Section 5: Conclusion} - Summary of key findings, their implications, suggestions for future research, and areas for improvement.
    \item \textbf{Section 6: Appendix} - Stores all figures relevant to the work that would otherwise disturb the above section's flow of reasoning.
\end{itemize}
